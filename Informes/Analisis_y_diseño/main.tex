\documentclass{article}
\usepackage[utf8]{inputenc}
\usepackage[spanish]{babel}
\usepackage{listings}
\usepackage{graphicx}
\graphicspath{ {images/} }
\usepackage{cite}

\begin{document}

\begin{titlepage}
    \begin{center}
        \vspace*{1cm}
            
        \Huge
        \textbf{Ideas para el proyecto final - Informatica II.}
            
        \vspace{0.5cm}
        \LARGE
            
        \vspace{1.5cm}
            
        \textbf{Luis Miguel Gil Rodriguez.}
        \\
        \textbf{Maverick Sosa Tobon.}
        \vfill
        \vspace{0.8cm}
            
        \Large
        Despartamento de Ingeniería Electrónica y Telecomunicaciones\\
        Universidad de Antioquia\\
        Medellín\\
        Marzo de 2021
            
    \end{center}
\end{titlepage}
\tableofcontents
\newpage
\section{Sección introductoria} \label{intro}
En este documento, podremos encontrar el diseño y modelación de objetos para el juego se se va a desarrollar.
\\
\\
Con el objetivo de cumplir co todos los requerimientos, se replantea la idea del juego. Contara con la misma trama, pero ahora, el personaje principal no se movera a lo largo del eje Y de la escena, sino a lo ancho del eje X, para poder facilitar observar con mas detalle los fenomenos fisicos.
\section{Diseño.}
\subsection{Modelaminto de Objetos.}
\subsubsection{Manejo de usuarios.}
\begin{itemize}
    \item Clase que se encarga de administrar y controlar el sistema de logueo de usuarios.
    \item Formulario CRUD:
    \begin{enumerate}
        \item Create: Crear usuarios
        \item Read: Leer usuarios.
        \item Update: Actualizar la información  de los usuarios cuando pase satisfactoriamente el mundo.
        \item Delete: Borrar usuarios.
    \end{enumerate}
    \item Va a ser la primera interfaz gráfica que aparezca, interactúa con el archivo de texto que contiene la información de los usuarios. En esta especie de menú principal se solicitará un usuario y contraseña, por lo cual, la clase debe estar en capacidad de leer dicho archivo y buscar en el mismo el usuario que se ingresó. Si existe y la contraseña ingresada coincide con la contraseña registrada en la base de datos del juego, se procederá a darle la bienvenida al usuario y enviarlo de inmediato al mundo en el que se encuentra actualmente. 
\end{itemize}

\subsubsection{Personaje Principal}
\begin{itemize}
    \item Genera el personaje principal del juego, se encarga del movimiento del mismo a lo largo del mundo.
    \item Se encarga de almacenar las posiciones en X e Y del personaje principal.
    \item Realiza el movimiento del personaje en la escena.
    \item El jugador principal podrá saltar en la escena para evitar los diferentes obstáculos el cual será modelado por medio del sistema físico denominado  movimiento parabólico.
    \item Contara con X cantidad de vidas para completar los mapas.
    \item Contara con un tiempo especifico para que se complete cada mapa.
\end{itemize}
\subsubsection{Enemigos.}
\begin{itemize}
    \item Genera los enemigos más simples: autos enemigos, obstáculos.
    \item Los enemigos en general, dependiendo de su tipo, tendrán uno u otro movimiento.
    \item Sus movimientos serán independientes del personaje, pero podrían verse afectados por algo que haga el personaje como un disparo por ejemplo.
    \item Dependiendo del nivel el que se encuentre el usuario, los enemigos avanzaran mas rapido.
    \item Se destruiran cuando un disparo del personaje principal, colisione con un enemigo.
    \item Cuando el personaje principal destruya uno de los enemigos le dara X cantidad de puntaje.
\end{itemize}
\subsubsection{Jefes Finales.}
\begin{itemize}
    \item Objeto con movimiento pendular (posiblemete bola de demolición) el cual va a ir mermando sus oscilaciones gradualmente gracias a la constante de fricción del viento.
    \item Se vencera de dos maneras:
    \begin{enumerate}
        \item El péndulo se quede quieto.
        \item Aguantará X disparos del jugador principal. (Atributo privado del objeto).
        Conforme el jugador vaya avanzando en los mundos, el jefe final aguantara mas disparos.
    \end{enumerate}
    \item Cuando el personaje principal logre destruir al jefe final de cada mundo, este objeto será destruido y se pasará de mundo.
\end{itemize}
\subsubsection{Balas.}
\begin{itemize}
    \item Son las que usa el personaje para eliminar a los enemigos, deshacerse de los obstáculos y enfrentar a los jefes. 
    \item Con este objeto se pretende aplicar el concepto de conservación  del momento lineal cuando impacte con los enemigos.
    \item Tienen una masa (esto para poder aplicar conservación del momento). 
    \item Tiene una velocidad y posición inicial.
    \item Tendran un movimiento lineal y tendran una cierta cantidad de rango de daño.
\end{itemize}
\subsection{Cronograma.}
El cronograma se podra econtrar especificamente en el archivo "cronograma.xlsx", el cual se encuentra dentro del repositorio.
\bibliographystyle{IEEEtran}
\bibliography{references}
\end{document}

